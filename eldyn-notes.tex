\documentclass[12pt]{article}


\usepackage[russian]{babel}
\usepackage[utf8]{inputenc}
\usepackage{fullpage}
\usepackage{mathtext}
\usepackage[intlimits]{amsmath}
\usepackage{mymathutils}

\newenvironment{remark}[1][Ремарка]{\par\noindent\rule{\textwidth}{.1pt}\par\textbf{#1}\par}{\par\noindent\rule{\textwidth}{.1pt}\par}
\newenvironment{problem}[1][]{\par\textbf{Задача: #1}\par}{\par}

\newcommand{\pdiff}[3][]{\frac{\partial^{#1} #2}{\partial #3^{#1}}}
\newcommand{\fdiff}[3][]{\frac{d^{#1} #2}{d #3^{#1}}}

\begin{document}
	%\section*{Введение}
%\section*{Литература}

\section{Уравнения Максвелла.}
В основе электродинамики лежат уравнения Максвелла. В дифференциальной форме они записываются следующим образом:
\begin{align*}
	\div{\vec{D}} &= \rho &
	\div{\vec{D}} &= 4\pi\rho \\
	\div{\vec{B}} &= 0 &
	\div{\vec{B}} &= 0 \\
	\rot{\vec{E}} &= -\pdiff{\vec{B}}{t} &
	\rot{\vec{E}} &= -\frac{1}{c}\pdiff{\vec{B}}{t} \\
	\rot{\vec{H}} &= \vec{j}(\vec{r}) + \pdiff{\vec{D}}{t} &
	\rot{\vec{H}} &= \frac{4\pi}{c}\vec{j}(\vec{r}) + \frac{1}{c}\pdiff{\vec{D}}{t}
\end{align*}

Интегральная форма этих же уравнений:
\begin{align*}
	\oint_{s} \vec{D}\cdot d\vec{\sigma} &= Q &
	\oint_{s} \vec{D} \cdot d\vec{\sigma} &= 4\pi Q \\
	\oint_{s} \vec{B} \cdot d\vec{\sigma} &= 0 &
	\oint_{s} \vec{B} \cdot d\vec{\sigma} &= 0 \\
	\oint_{l} \vec{E} \cdot d\vec{l} &= -\fdiff{}{t}\int_s \vec{B}\cdot d\vec{\sigma} &
	\oint_{l} \vec{E} \cdot d\vec{l} &= -\frac{1}{c}\fdiff{}{t}\int_s \vec{B}\cdot d\vec{\sigma} \\
	\oint_{l} \vec{H} \cdot d\vec{l} &= I + \fdiff{}{t}\int_s \vec{D} \cdot d\vec{\sigma} &
	\oint_{l} \vec{H} \cdot d\vec{l} &= \frac{4\pi}{c}I + \frac{1}{c}\fdiff{}{t}\int_s \vec{D} \cdot d\vec{\sigma}
\end{align*}

Дифференциальная форма уравнений получается из интегральной путём применения теорем Гаусса --- Остроградского и Стокса.

Стоит отметить, что уравнения Максвелла описывают лишь распространение волн, но не их генерацию.

\subsection{Материальные уравнения}
Для того, чтобы решить задачу электродинамики --- нахождение полей \(\vec{E}(\vec{r}, t)\), \(\vec{H}(\vec{r}, t)\), \(\vec{D}(\vec{r}, t)\) и \(\vec{B}(\vec{r}, t)\), эту систему из четырёх уравнений необходимо дополнить также материальными уравнениями, например,
\begin{align*}
	\vec{D} &= \epsilon_0\epsilon\vec{E} &
	\vec{D} &= \epsilon\vec{E} \\
	\vec{B} &= \mu_0\mu\vec{H} &
	\vec{B} &= \mu\vec{H} \\
	\vec{j} &= \sigma\vec{E} &
	\vec{j} &= \sigma\vec{E}.
\end{align*}
Также иногда требуется привлечь силу взаимодействия зарядов:
\begin{align*}
	\vec{F} &= q\left(\vec{E} + [\vec{v} \times \vec{B}]\right) & 
	\vec{F} &= q\left(\vec{E} + \frac{1}{c}[\vec{v} \times \vec{B}]\right).
\end{align*}

Сделаем несколько замечаний о материальных уравнениях. Что вообще микроскопически представляет из себя
диэлектрическая проницаемость \(\epsilon\)? Она описывает поляризацию --- переориентацию или изменение величины
 дипольных моментов, присутствующих в среде. Поля этих диполей вносят в
электрическое поле добавку, описываемую поляризуемостью \(\chi\):
\begin{align*}
	\vec{D} = \epsilon_0\epsilon \vec{E} = \epsilon_0(1 + \chi)\vec{E},
\end{align*}
при этом наличие какого-либо физического смысла у постоянной \(\epsilon_0\),
отсутствующей в системе СГС, является открытым вопросом.

Поляризуемость описывает поляризацию, т.\,е., поворот дипольных моментов:
\begin{align*}
	\epsilon_0 \chi \vec{E} = \vec{P} = \frac{1}{V}\sum_{i} \vec{p}_i.
\end{align*}

С точки зрения того, что выступает в роли этих дипольных моментов, 
можно выделить следующие виды поляризации:
\begin{itemize}
	\item дипольная: как правило, в данном случае речь идёт о веществах, состоящих из полярных молекул.
	Основной вклад в поляризацию в таких веществах обеспечивается за счёт поворота молекул вдоль поля.
	Время релаксации к поляризации в таких веществах составляет порядка \(10^{-10}~\text{с}\). Оценить его
	можно следующим образом: на поляризации воды работают микроволновые печи, следовательно, их частоту
	подбирают такой, чтобы молекулы в среднем успевали повернуться за полупериод волны (на соответствующей
	частоте должен наблюдаться максимум поглощения). Частота бытовой СВЧ-печи равна 2450~МГц. Следовательно, 
	\(\tau_{\text{H}_2\text{O}} \sim 1 / (2.45\cdot10^{9}~\text{с}^{-1}) = 4\cdot10^{-10}~\text{c}\);
	\item ионная: в данном случае речь идёт о кристаллах. В них дипольные моменты не могут поворачиваться,
	но ионы могут смещаться из положений равновесия в кристаллической решётке. Время релаксации порядка
	\(10^{-13}~\text{с}\);
	\item электронная: в роли дипольных моментов выступают атомы. Электрическое
	поле может деформировать их электронные оболочки. Время релаксации данного вида
	поляризации составляет порядка \(10^{-14}\)--\(10^{-15}\)~с.
\end{itemize}
Ориентация частиц, собственные частоты которых совпадают с частотой <<внешнего>> поля, называется резонансной поляризацией.

Помимо упомянутых типов поляризации, можно отметить спонтанную поляризацию (пироэлектричество),
сегнетоэлектричество, пьезоэлектричество\ldots

То же можно записать и для магнитной поляризации, но поворачиваться будут уже не дипольные моменты,
а магнитные:
\begin{align*}
	\vec{B} = \mu_0\mu\vec{H} = \mu_0(1 + &\gamma)\vec{H}, \\
	\mu_0&\gamma\vec{H} = \vec{M} = \frac{1}{c}\sum_i \vec{m}_i.
\end{align*}

\subsection{Замечание о киральных средах}
Можно предположить, что электрическая индукция зависит только от электрического поля,
а магнитная --- только от магнитного, однако это не так.%
\footnote{Получается, Бредов, Румянцев и Топтыгин неправы? Надо посмотреть, они, вроде, писали что-то на этот счёт.}
Примером сред, в которых такое описание не работает, являются киральные среды [здесь надо бы дать их определение и какой-нибудь обзор на четверть страницы.]

В частном случае киральных сред --- в диадных средах, --- материальные уравнения записываются следующим образом:
\begin{align*}
	\vec{D} &= \epsilon_0\epsilon E - i\eta\vec{H} \\
	\vec{H} &= \mu_0\mu\vec{B} + i\eta{B},
\end{align*}
где \(\eta\) называется элементом киральности.

Одно из особых свойств киральных сред демонстрируется в следующем простом эксперименте.
Прозрачный для радиоволн ящик заполняется левыми и правыми пружинами, в определённом соотношении.
К ящику подводится антенна или волновод, из которого приходит поляризованное(?) излучение от источника.
После прохода излучением ящика с пружинами, наблюдается поворот угла поляризации на угол, зависящий от соотношения
левых и правых пружин.



	\section{Решение уравнений Максвелла}
REFINE!!!
\subsection{Принцип перестановочной двойственности}
Запишем уравнения Максвелла, разделив предварительно токи и заряды на внутренние
и сторонние:
\begin{align*}
	\div{\vec{D}} &= \rho + \rho_{\text{ст.}} &
	\div{\vec{D}} &= 4\pi(\rho + \rho_{\text{ст.}}) \\
	\div{\vec{B}} &= 0 &
	\div{\vec{B}} &= 0 \\
	\rot{\vec{E}} &= -\pdiff{\vec{B}}{t} &
	\rot{\vec{E}} &= -\frac{1}{c}\pdiff{\vec{B}}{t} \\
	\rot{\vec{H}} &= \vec{j}(\vec{r}) + \vec{j_{\text{ст.}}}(\vec{r}) + \pdiff{\vec{D}}{t} &
	\rot{\vec{H}} &= \frac{4\pi}{c}\left(\vec{j}(\vec{r}) + \vec{j_{\text{ст.}}}(\vec{r})\right) + \frac{1}{c}\pdiff{\vec{D}}{t}
\end{align*}

Принцип перестановочной двойственности:
можно выполнить замену:
\begin{align*}
  \vec{E} &\to \vec{H}, &
  \sigma_{\text{эл.}} &\to \sigma_{\text{магн.}}, &
  \epsilon_0\epsilon &\to -\mu_0\mu, &
  \vec{j_{\text{эл.}}} &\to \vec{j_{\text{магн.}}}.
\end{align*}
Тогда получим уравнения для магнитных зарядов и токов.

\subsection{Подходы к решению уравнений Максвелла}
<<В прямую>> уравнения Максвелла не решаются. Существует четыре способа
сведения их к более удобной для решения форме:
\begin{enumerate}
  \item волновое уравнение для электрического и магнитного полей;
  \item волновое уравнение для скалярного и векторного потенциалов;
  \item волновое уравнение для электрического и магнитного векторов Герца;
  \item уравнение первого порядка для вектора Римана --- Зилберштейна.
\end{enumerate}
Для получения квантовой теории квантование удобно применять к последним двум
представлениям.

\subsection{Волновое уравнение для полей}
Из уравнений Максвелла легко получить волновое уравнение
\begin{align*}
  \nabla^2\vec{E} - \frac{\epsilon_0\epsilon\mu_0\mu}{c^2} \pdiff[2]{\vec{E}}{t} = 0.
\end{align*}
Простейшим решением данного уравнения является плоская волна:
\begin{align*}
  \vec{E} = \vec{E_0}\exp\left[i(\omega t - \vec{\kappa}\vec{r})\right].
\end{align*}
Фазовая скорость волны \(v_{\phi} = \omega/\kappa\):
\begin{align*}
  \frac{\epsilon_0\epsilon\mu_0\mu}{c^2} = \frac{1}{v_{\phi}^2}, 
\end{align*}
следовательно,
\begin{align*}
  v_{\phi} = \frac{c}{\sqrt{\epsilon_0\mu_0\epsilon\mu}} = \frac{c}{n}.
\end{align*}
Фаза постоянна в пространстве, когда \(\vec{\kappa}\vec{r} = \const\), что
являет собой уравнение плоскости. Поэтому такая волна --- плоская.

Другим решением является сферическая волна:
\begin{align*}
  \vec{E}(\vec{r}, t) = \frac{\vec{E}}{\abs{\vec{r}}} \exp\left[i(\omega
  t - \vec{\kappa}\vec{r})\right].
\end{align*}

// Получить закон сохранения заряда и волновые уравнения с \(\rho\)
и \(\vec{j} \neq 0\).

\subsection{Векторный и скалярный потенциалы}
Скалярный потенциал \(\phi\) и векторный потенциал \(\vec{A}\) вводятся
следующим образом:
\begin{align*}
  \vec{E} &= -\grad{\phi} - \pdiff{\vec{A}}{t}, \\
  \vec{B} &= \rot{\vec{A}}.
\end{align*}

Для них можно получить волновые уравнения
\begin{align*}
  \nabla^2 A - \frac{\epsilon\mu}{c^2} \pdiff[2]{\vec{A}}{t} &= -\mu_0\mu
  \vec{j_{\text{ст.}}}, \\
  \nabla^2\phi - \frac{\epsilon\mu}{c^2} \pdiff[2]{\phi}{t} &=
  -\frac{\rho_{\text{ст.}}}{\epsilon_0\epsilon}.
\end{align*}
// калибровка?

//Доказать калибровочную инвариантность.

\subsection{Волновые уравнения для векторов Герца}
Вводят два вектора Герца: электрический и магнитный.
// вывести это всё и написать здесь.
\subsection{Вектор Римана --- Зильберштейна}
// привести всё это здесь.


	\subsection{Снова о методах}
Итак, если не рассматривать вектор Римана --- Зильберштейна, электрическое и
магнитное поля можно получить из уравнений Максвелла, преобразовав их к волновому уравнению.
Его можно записать для:
\begin{itemize}
	\item \(\vec{E}, \vec{H}\);
	\item \(\phi, \vec{A}\);
	\item \(\Gamma^{\text{эл.}}, \Gamma^{\text{магн.}}\).
\end{itemize}
Подход через векторы Герца привлекателен тем, что сразу даёт разделение на TE- и TM-поляризацию.
Это и есть общий матфизический подход к уравнениям Максвелла; всё остальное --- уже комбинаторика.

\subsection{Символический метод}
Если с самого начала искать решение в виде плоских монохроматических волн, или же, если угодно, подвергнуть уравнения
Максвелла преобразованию Фурье, то это приведёт к следующей замене, имеющей громкое название <<\emph{символический метод}>>:
\begin{align*}
	\pdiff{}{t} &\to i\omega &
	\nabla &\to i\vec\kappa.
\end{align*}
%Волновое уравнение при этом преобразуется в уравнение Гельмгольца:
%Из второго уравнения при этом л
Уравнение Гельмгольца (как возникает?)
\begin{align*}
	\nabla^2\vec{E} + \kappa^2\vec{E} = 0.
\end{align*}

\subsection{Плоские и монохроматические волны: применимость приближения}
В предыдущем параграфе мы решили искать решение уравнение Максвелла в виде
плоских монохроматических волн. Тем не менее, очевидно, что ни идеально плоских,
ни идеально монохроматических волн в природе не существует и существовать не
может, так как такая волна была бы бесконечной в пространстве либо во времени
соответственно. Поэтому говорят о квазиплоских и квазимонохроматических волнах.
Для начала, дадим определение этим понятиям.

\emph{Квазимонохроматическое излучение} --- излучение, ширина спектра которого
много меньше частоты (\(\Delta\omega/\omega \ll 1\)).

Без применения специальных мер невозможно получить спектральную ширину линии
уже, чем естественная ширина, так как последняя является характеристикой атома,
а источники излучения из этих самых атомов состоят.

// получить и привести выражение для естественной ширины линии

Ширина линии одномодового лазера may be as small as \(~1~\text{МГц}\).
\begin{remark}
При этом методами нелинейной спектроскопии можно достигнуть на два порядка
меньших значений.
\end{remark}
Факторы, влияющие на уширение линии: столкновительный, допплеровский...

//здесь надо привести таблицу ширин линий источников: лазера, ВЧ-диода,
клистрона...

Теперь обсудим пространственные характеристики. Реальное излучение всегда
испускается источником конечного размера, поэтому его волновой фронт всегда
изогнут. Однако на большом расстоянии от источника волновые фронты имеют столь
большой радиус кривизны, что являются почти плоскими. Осталось пояснить, что
значит <<почти>>. Разумно сравнивать прогиб волнового фронта с длиной волны.
Таким образом, \emph{квазиплоская волна} --- это волна, прогиб волнового фронта
которой меньше или порядка длины волны. Стоит заметить, что прогиб волнового
фронта зависит от площади, на которой рассматривается излучение, поэтому 
квазиплоское в некоторой площади излучение может перестать быть квазиплоским,
если его рассматривать в более широкую апертуру.

Стоит рассмотреть и другой предельный случай --- сферическую волну. Волна
считается сферической, если размер источника меньше или порядка длины волны.
// чё-то тут нечисто

\section{Мнимая часть диэлектрической проницаемости. Модель Дебая}
\begin{align*}
  \epsilon = \epsilon' - i\epsilon''
\end{align*}
\begin{align*}
  \epsilon = \epsilon_{\infty} + \frac{\epsilon_{s}
  - \epsilon_{\infty}}{1 + i\omega\tau}
\end{align*}
Данная формула является чисто феноменологической, и не включает в себя механизма
поляризации.

Из комплексной диэлектрической проницаемости радиофизики рассматривают тангенс
угла потерь:
\begin{align*}
  \tg{\phi} = \frac{\epsilon'}{\epsilon''},
\end{align*}
а оптики -- коэффициенты преломления и экстинкции.


	\section{Временная дисперсия на языке плоских волн. Формула Релея}
Как было показано в предыдущих параграфах, фазовая скорость плоской волны
равна
\begin{align*}
  v_{\phi} = \frac{\omega}{\kappa} = \frac{c}{n}
  = \frac{c}{\sqrt{\epsilon}{\mu}}.
\end{align*}

Это скорость движения фронта плоской монохроматической волны. Что же произойдёт
если в среду запустить волновой пакет --- набор плоских волн с различными
частотами? Будет наблюдаться \emph{временная дисперсия} --- расплывание
волнового пакета.

Для анализа прохождения волнового пакета через среду вводится понятие
\emph{групповой скорости}:
\begin{align*}
  v_{\text{гр.}} \equiv \left(\pdiff{\omega}{\kappa}\right)_{\kappa = \kappa_0},
\end{align*}
где волновой вектор \(\kappa_0\) соответствует центру спектра волнового пакета
(?).

Легко формально показать, что
\begin{align*}
  v_{\text{гр.}} = v_{\phi} + \kappa \pdiff{v_{\phi}}{\kappa}
\end{align*}
Эту формулу и называют формулой Релея. Но можно пойти дальше и перевести всё на
язык оптики, в котором работают с показателем преломления и длиной волны. Тогда
формула будет выглядеть следующим образом:
\begin{align*}
  v_{\text{гр.}} &= v_{\phi} - \lambda \pdiff{v_{\phi}}{\lambda} \\
  \lambda\pdiff{v_{\phi}}{\lambda} &= -\frac{c}{n^2}\pdiff{n}{\lambda}.
\end{align*}

Здесь мы получаем три случая:
\begin{itemize}
  \item \(v_{\text{гр.}} = v_{\phi}\), или же \(\pdiff{n}{\lambda} = 0\) ---
    отсутствие дисперсии;
  \item \(v_{\text{гр.}} < v_{\phi}\), или же \(\pdiff{n}{\lambda} < 0\) ---
    нормальная дисперсия;
  \item \(v_{\text{гр.}} > v_{\phi}\), или же \(\pdiff{n}{\lambda} > 0\) ---
    аномальная дисперсия.
\end{itemize}

// привести картинку \(\varepsilon\) с аномальной дисперсией.


\section{Пространственные частоты. Пространственная фильтрация. Параксиальное
приближение}
Из выражения
\begin{align*}
  \frac{c}{n} = \frac{\omega}{\kappa}
\end{align*}
легко получить
\begin{align*}
  \frac{\kappa^2_x + \kappa^2_y + \kappa^2_z}{\epsilon\mu}
  = \frac{\omega^2}{c^2}.
\end{align*}
В этом выражении уже видна некоторая аналогия между волновыми векторами и частотами.
Это побуждает рассматривать волновые векторы как <<пространственные частоты>>.

В предыдущем параграфе рассматривался волновой пакет из плоских волн; в общем же
случае у волн в волновом пакете различается не только модуль волнового вектора
(что соответствует различным частотам), но и его направление. С этим связано
понятие \emph{пространственной дисперсии}: пусть на рассеивающую среду падает
плоская волна. В результате рассеяния появляются вторичные волны,
распространяющиеся в направлениях, отличных от изначального. То есть, у них
изменились направления волнового вектора. Такое изменение и называется
пространственной дисперсией.

С этим же тесно связано и понятие пространственной фильтрации: если приёмник
с некоторой конечной апертурой установить слишком далеко от рассеивающей среды,
то слишком сильно рассеянные волны просто не попадут в эту апертуру --- они будут
отфильтрованы. Стоит заметить, что если мы не хотим потерять всей информации
о волновом пакете (то есть, и частотного, и пространственного спектра), то
пространственная фильтрация нежелательна, так как ведёт к потере части
рассеянных волн.

\subsection{Математический аппарат пространственной дисперсии}
Выберем ось \(z\) вдоль направления распространения волнового пакета как целого.
К полю можно применить преобразование Фурье по компонентам волнового вектора,
перпендикулярным направлению распространения:
\begin{align*}
  \vec{E}(\kappa_x, \kappa_y, z=0) &= \frac{1}{4\pi^2}\iint_{-\infty}^{+\infty}
  \vec{E}(x, y, z=0) \exp\left[i(-\kappa_x x - \kappa_y y)\right] dxdy \\
  \vec{E}(x, y, z=0) &= \iint_{-\infty}^{+\infty} \vec{E}(\kappa_x, \kappa_y, z=0)
  \exp\left[i(\kappa_x x + \kappa_y y)\right] d\kappa_xd\kappa_y. 
\end{align*}
Если теперь в таком виде подставить электрическое поле в уравнение Гельмгольца
\begin{align*}
  \nabla^2\vec{E} + \kappa_z^2\vec{E} = 0,
\end{align*}
то получим решение в виде
\begin{align*}
  \vec{E}(\kappa_x, \kappa_y, z) = \vec{E}(\kappa_x, \kappa_y, z=0)\exp(\pm
  i\kappa_z z).
\end{align*}
Переходя к Фурье-прообразу, получаем:
\begin{align*}
  \vec{E}(x, y, z) = \iint_{-\infty}^{+\infty} \vec{E}(\kappa_x, \kappa_y, z=0)
  \exp\left[i(\kappa_xx + \kappa_yy + \kappa_zz)\right] d\kappa_xd\kappa_y.
\end{align*}
Таким образом, мы получили спектр по \(\kappa_x\) и \(\kappa_y\).

\subsection{Замечание о разрешении изображений}
В классической оптике разрешение ограничивается критерием Релея --- Джинса,
т.\,е., не точнее \(\frac{\lambda}{2}\). Однако если располагать информацией о:
\begin{itemize}
  \item частотном спектре \(\vec{E}(\omega)\);
  \item пространственном спектре \(\vec{E}(\kappa_x, \kappa_y)\);
  \item эванесцентных модах,
\end{itemize}
--- то этот предел разрешения можно преодолеть (см., например, работы Белова).

\subsection{Параксиальное приближение}
Рассмотрим поле в дальней зоне: \(r \gg \lambda\). Квазиплоское в гауссовом
приближении поле можно разложить по пространственным частотам. При этом
\begin{align*}
  \kappa^2 = \kappa^2_x + \kappa^2_y + \kappa^2_z,
\end{align*}
откуда
\begin{align*}
  \kappa_z = \sqrt{\kappa^2 - \kappa^2_x - \kappa^2_y} \simeq \kappa
  - \frac{\kappa_x^2 + \kappa_y^2}{2\kappa}.
\end{align*}
Применённое здесь разложение в ряд и составляет суть параксиального приближения.
В параксиальном приближении получается критерий Релея --- Джинса.
\begin{remark}
  Дифракционный угол одномодового лазера --- \(\phi \simeq 1.22\lambda/D\).
\end{remark}

\section{Когерентность}
Поле любого реального источника необходимо представлять в виде
\begin{align*}
  \vec{E} = \vec{E_0} \exp\left[i(\omega t - \vec{\kappa}\vec{r}
  + \phi(t))\right],
\end{align*}
где \(\phi(t)\) --- случайная фаза. 
\begin{remark}[Ремарка от меня]
  Если хочется порассуждать о когерентности в радиотехнике, то там выглядит
  более разумным рассуждать не о случайной фазе, а о случайном изменении частоты
  --- например, из-за теплового расширения резонатора. Но это всё лирика.
\end{remark}
На тему этой случайной фазы можно рассуждать много и долго\footnote{Например, эргодическая
гипотеза, которой все пользуются --- никакого обоснования, кроме как ,,ну,
наверное, это так'' под собой не имеет. Кто жаждет подробностей --- они есть
в книге <<Когерентность света>> Яна Перины.}, но мы здесь рассмотрим лишь одно
предположение: что случайное изменение фазы --- это \emph{стационарный случайный
процесс}. Стационарность здесь означает однородность во времени всех усредняемых
параметров:
\begin{align*}
  \forall t_1, t_2 \quad \frac{1}{\tau}\int_{t_1}^{t_1 + \tau} f(t)dt = 
  \frac{1}{\tau}\int_{t_2}^{t_2 + \tau} f(t)dt
\end{align*}

Итак, рассмотрим опыт Юнга --- двухлучевую интерференцию. Поля от двух
точечных источников (или узких щелей) складываются и, возможно, образуют на
экране интерференционную картину. Запишем для этого случая квадрат поля, считая
его, для простоты, скалярным:
\begin{align*}
  \abs{E_1 + E_2}^2 = \abs{E_1}^2 + \abs{E_2}^2 + 2\abs{E_1E_2}
\end{align*}
При этом интенсивность, регистрируемая
квадратичным детектором, будет выражаться следующим образом:
\begin{align*}
  \frac{1}{\tau}\int_{t_0}^{t_0 + \tau} \abs{E_1 + E_2}^2 dt
  = \frac{1}{\tau}\int_{t_0}^{t_0 + \tau} \abs{E_1}^2dt 
  + \frac{1}{\tau}\int_{t_0}^{t_0 + \tau} \abs{E_2}^2dt 
  + 2\frac{1}{\tau}\int_{t_0}^{t_0 + \tau} \abs{E_1E_2}dt.
\end{align*}
Введя измеряемые интенсивности
\begin{align*}
  I_i = \frac{1}{\tau}\int_{t_0}^{t_0 + \tau} \abs{E_i}^2 dt,
\end{align*}
можем переписать это в виде
\begin{align*}
  I_{1+2} = I_1 + I_2 + 2\sqrt{I_1I_2}\abs{\gamma_{21}(\tau)},
\end{align*}
где \(\gamma_{21}(\tau)\) (далее --- просто \(\gamma(\tau)\)) --- степень
когерентности, определённая, как
\begin{align*}
  \gamma(\tau) = \cfrac{
  \left[\frac{1}{\tau}\int_{t_0}^{t_0 + \tau} \abs{E_1E_2}dt \right]
  }{
  \sqrt{
    \left[\frac{1}{\tau}\int_{t_0}^{t_0 + \tau} \abs{E_1}^2dt\right]
    \left[\frac{1}{\tau}\int_{t_0}^{t_0 + \tau} \abs{E_2}^2dt\right] 
  }
  }.
\end{align*}
Как нетрудно заметить, \(0 \le \abs{\gamma(\tau)} \le 1\), причём
\(\gamma(\tau) = 0\) соответствует полностью некогерентному свету (складываются
интенсивности), а \(\gamma(\tau)\) --- полностью когерентному (складываются
амплитуды).

Экспериментальным способом определения когерентности является опыт Юнга.
// расписать это здесь как следует

// Нужно упомянуть теоремы Винера --- Хинчина и ван Циттера --- Цернике, но я не
помню, с какой целью

// Рассказать кулстори про опподизацию и геометрическую дифракцию





	\section{Предельные случаи в электродинамике}
\begin{itemize}
  \item \(L \ll \lambda\) --- квазистатическая область. Считается
  канонизированной и полностью решённой.
  \item \(L \sim \lambda\) --- резонансная область. К её решению выделяют три
  основных подхода:
  \begin{itemize}
    \item метод Фурье (разделение переменных в волновом уравнении);
    \item метод запаздывающих потенциалов;
    \item метод векторного и скалярного интегралов Кирхгофа (хорошо описывает
    поведение волны за экраном с апертурой; в случае, если этот экран
    металлический, из-за токов возникают сложности).
  \end{itemize}
  \item \(L \gg \lambda\) --- квазиоптическая область. Основные методы решения
  -- геометрическая оптика (без дифракции); уравнение эйконала и геометрическая
  теория дифракции.
\end{itemize}

\section{Энергетические соотношения}
Итак, допустим, мы получили решение уравнений Максвелла, например, в виде
плоской волны:
\begin{align*}
  \vec{E} &= \vec{E_0} \exp\left[i(\omega t - \vec{\kappa}\vec{r}
  + \phi(t))\right].
\end{align*}
Возникает вопрос: как получить его экспериментальное подтверждение? Для этого
нужно измерить:
\begin{itemize}
  \item амплитуду \(\abs{\vec{E_0}}\);
  \item поляризацию (направление \(\vec{E_0}\);
  \item частоту \(\omega\);
  \item длину волны \(\lambda = \frac{2\pi}{\kappa}\);
  \item фазу \(\phi(t)\).
\end{itemize}
При этом наиболее точно мы можем измерить амплитуду; поляризацию --- примерно
в \(10^4\) раз грубее (методы эллипсометрии); частоту и длину волны --- примерно
с той же точностью, что и поляризацию (определяется имеющимися стандартами
частоты и длины); фазу измерять мы вообще не можем --- только разность фаз.

Так как амплитуда измеряется наиболее точно, для экспериментального
подтверждения уравнений Максвелла важно получить энергетические соотношения.

\begin{remark}[Об опыте Лебедева с пондеромоторной силой (передача импульса от
света)]
  Этот опыт был проделан более 100~лет назад, но при этом до сих пор ни
  в школьные, ни в студенческие лабы его не включают. Это печально,
  потому что опыт очень показательный и не требует сложного оборудования.
\end{remark}

\subsection{Теорема Умова --- Пойнтинга}
Запишем два уравнения Максвелла, для простоты опустив токи:
\begin{align*}
  \left[\vec{\nabla} \times \vec{H}\right] &= \pdiff{\vec{D}}{t} \\
  \left[\vec{\nabla} \times \vec{E}\right] &= -\pdiff{\vec{B}}{t}
\end{align*}
Домножив уравнения скалярно на \(\vec{E}\) и \(\vec{H}\) соответственно,
вычтем их друг из друга:
\begin{align*}
  \vec{E}\cdot\left[\vec{\nabla} \times \vec{H}\right] -
  \vec{H}\cdot\left[\vec{\nabla} \times \vec{E}\right] &= 
  \vec{E}\pdiff{\vec{D}}{t} - \vec{H}\pdiff{\vec{B}}{t}.
\end{align*}
После этого воспользуемся векторным тождеством
\begin{align*}
  \vec{E}\cdot\left[\vec{\nabla} \times \vec{H}\right]
  -
  \vec{H}\cdot\left[\vec{\nabla} \times \vec{E}\right]
  &= 
  -\div\left[\vec{E}\times\vec{H}\right].
\end{align*}
С другой стороны, воспользуемся тем, что
\begin{align*}
  \epsilon \pdiff{}{t}\vec{E}^2 = \epsilon \pdiff{E^2}{E}\pdiff{E}{t}
  = 2 E \pdiff{(\epsilon E)}{t},
\end{align*}
то есть,
\begin{align*}
  \vec{E}\pdiff{\vec{D}}{t} = \frac{1}{2}\epsilon\pdiff{E^2}{t}, 
\end{align*}
и, аналогично,
\begin{align*}
  \vec{H}\pdiff{\vec{B}}{t} = \frac{1}{2}\mu\pdiff{H^2}{t}.
\end{align*}
Пользуясь полученными равенствами, получаем
\begin{align*}
  \pdiff{}{t}\left[ \frac{\epsilon E^2 + \mu H^2}{2} \right]
  +\div\left[\vec{E}\times\vec{H}\right] = 0.
\end{align*}
Это закон сохранения в дифференциальной форме. Речь идёт о сохранении энергии,
отсюда плотность энергии
\begin{align*}
  w = 
  \frac{\epsilon E^2 + \mu H^2}{2},
\end{align*}
а плотность потока энергии (вектор Умова --- Пойнтинга) ---
\begin{align*}
  \vec{S} = \left[\vec{E} \times \vec{H}\right].
\end{align*}
Закон сохранения легко перевести из дифференциальной формы в интегральную,
проинтегрировав по некоторому замкнутому объёму, и, применив теорему Гаусса ---
Остроградского, перейти от интеграла дивергенции по объёму к интегралу по
поверхности:
\begin{align*}
  \pdiff{}{t}\iiint_{V} w dV = 
  \iiint_{V} \div\vec{S} dV = 
  \iint_{\partial V} \vec{S} \cdot d\vec{\sigma}.
\end{align*}
Поток энергии через единичную область представляет собой интенсивность:
\begin{align*}
  I = 
  \iint_{1\times 1} \left[\vec{E} \times \vec{H}\right] d\vec{\sigma} = 
  \iint_{1\times 1} \left[\vec{E} \times
  \frac{\sqrt{\mu_0\mu}}{\sqrt{\epsilon_0\epsilon}}\vec{E}\right] d\vec{\sigma} = \frac{\abs{E}^2}{Z},
\end{align*}
где мы воспользовались соотношением для плоских волн
\begin{align*}
  \sqrt{\epsilon_0\epsilon}\vec{E} = \sqrt{\mu_0\mu}\vec{H}
\end{align*}
и ввели импеданс
\begin{align*}
  Z = \frac{\sqrt{\epsilon_0\epsilon}}{\sqrt{\mu_0\mu}}.
\end{align*}
Таким образом, для плоской волны \(\vec{E} = \vec{E_0}\cos{\omega t}\),
\begin{align*}
  I = \frac{E_0^2}{Z} \cos^2\omega t.
\end{align*}
Автор конспекта предлагает обратить внимание на сходство с законом Джоуля ---
Ленца:
\begin{align*}
  P = \frac{U_0^2}{Z},
\end{align*}
причём электрическое поле измеряется в \(\text{В}/\text{м}\), а интенсивность ---
в~\(\text{Вт}/\text{м}^2\).

В реальном эксперименте в оптическом диапазоне детектор не может уследить за
множителем \(\cos\omega t\), поэтому наблюдается усреднённое значение
интенсивности:
\begin{align*}
  I = \frac{1}{\tau}\int_{t_0}^{t_0 + \tau}\frac{E_0^2}{Z} \cos^2\omega
  t = \frac{E_0^2}{2Z}.
\end{align*}
Именно этой формулой обычно и пользуются. В системе СГС импеданс вакуума равен~1.
  

	\section{Модель Друде --- Лоренца --- Зоммерфельда}
В металлах диэлектрическая проницаемость описывается формулой Дебая:
\begin{equation}
  \varepsilon^{*} = \varepsilon_{\infty} 
-
\underbrace{
  \frac{\varepsilon_0 - \varepsilon_{\infty}}{1 + i\omega \tau}
}_{\text{резонансное поглощение}}
  -
\underbrace{
i\frac{\sigma}{\omega\varepsilon_0}
}_{\text{экранировка}}
\end{equation}

Для примера рассмотрим параметры трёх лучших металлов: Ag, Au, Cu.
В ИК-диапазоне у них наблюдается показатель преломления \(n < 1\).
Как это объяснить?

При построении теории оптики хороших металлов, следует иметь ввиду, что для
них \(\sigma \to \infty\) и \(R \to 1\). Для сравнения:
\(\sigma_{\text{бескислородн. Cu}} \sim 10^7\ldots10^8\).

\subsection{Отклик идеального газа электронов}
Будем рассматривать электронный газ в металле как идеальный.
Для построения модели Лоренца --- Друде --- Зоммерфельда проделываются следующие
шаги:
\begin{enumerate}
\item
Записываются силы, действующие на электрон --- сила трения \(m\gamma\dot{r}\)
и взаимодействие с электрическим полем волны \(eEe^{i\omega t}\).
\item
Из них получается уравнение 
\begin{equation}
  \ddot{r} + \gamma\dot{r} = E\frac{e}{m} e^{i\omega t}.
\end{equation}
Решение этого уравнения ---
\begin{equation}
  r(t) = -Ee^{i\omega t} \frac{e}{m} \frac{1}{\omega^2 - i\gamma\omega}.
\end{equation}
\item
Одиночный электрон обладает дипольным моментом \(e r(t)\); все электроны
в единице объёма вещества ---
\begin{equation}
  P = \frac{1}{V} \sum_i p_i = ner(t),
\end{equation}
где \(n\) --- концентрация электронов.
\item
Так как \(\vec{D} = \varepsilon_0 \vec{E} + \vec{P}\), 
\begin{equation}
  \varepsilon(\omega) = \frac{De^{i\omega t}}{Ee^{i\omega t}} = \varepsilon_0 + \frac{ner(t)}{Ee^{i\omega t}}
  =
  \varepsilon_0 - \frac{ne^2}{m}\frac{1}{\omega^2 - i\gamma\omega}.
\end{equation}
\end{enumerate}
Таким образом, получаем отклик в виде 
\begin{equation}
  \varepsilon(\omega) = \varepsilon_0 \left[1 - \frac{\omega_p^2}{\omega^2
  - i\gamma\omega}\right],
\end{equation}
где \(\omega^2_p = e^2n/m\varepsilon_0\) называется \emph{плазменной частотой}.

Для хороших металлов \(\gamma \ll \omega\), поэтому диэлектрическая
проницаемость для них имеет вид
\begin{equation}
  \varepsilon(\omega) = \varepsilon_0\left[1
  - \frac{\omega_p^2}{\omega^2}\right].
\end{equation}
При этом, как легко видеть, в области \(\omega > \omega_p\) наблюдается \(n
< 1\).

// Может, в начале про \(n < 1\) речь шла про ультрафиолет, а не ИК?

В качестве характеристики потерь можно ввести \emph{тангенс угла потерь}, равный
\begin{align*}
  \tg\varphi = \frac{\varepsilon''}{\varepsilon'}.
\end{align*}

Подробнее почитать можно в: Климове (наноплазмоника); Сарычев, Шалаев ---
<<Электродинамика метаматериалов>>.

\subsection{Смысл плазменной частоты}
Уравнению Максвелла \(\div \vec{D} = \varepsilon(\omega)\div\vec{E} = 0\)
можно удовлетворить двумя способами:
\begin{itemize}
  \item \(\div\vec{E} = 0\) --- поперечная волна;
  \item \(\varepsilon(\omega) = 0\) --- продольная волна.
\end{itemize}

Разберём эти два случая по-отдельности.
\paragraph{Поперечная волна}
Для поперечной волны уравнение Гельмгольца имеет вид
\begin{equation}
  \underbrace{
  -\vec{\kappa}
  (\vec{\kappa}\cdot\vec{E})
  }_{= 0}
  + \kappa^2 \vec{E}(\vec{\kappa}, \omega) = \varepsilon(\vec{\kappa}, \omega)
  \frac{\omega^2}{c^2} \vec{E}(\vec{\kappa}, \omega).
\end{equation}

Отсюда получаем дисперсионное соотношение:
\begin{equation}
  \omega \varepsilon(\vec{\kappa}, \omega) = c^2 \kappa^2.
\end{equation}

Если взять \(\varepsilon(\omega) = \varepsilon_0(1 - \omega_p^2 / \omega^2)\),
то получается дисперсионное соотношение поперечного поляритона:
\begin{equation}
  \omega^2 = \omega_p^2 + c^2\kappa^2.
\end{equation}

\paragraph{Продольная волна}
В случае продольной волны (\(\varepsilon(\omega, \vec{\kappa}) = 0\)),
пространственную дисперсию исключать нельзя. 

\end{document}
