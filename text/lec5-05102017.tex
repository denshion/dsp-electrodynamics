\section{Предельные случаи в электродинамике}
\begin{itemize}
  \item \(L \ll \lambda\) --- квазистатическая область. Считается
  канонизированной и полностью решённой.
  \item \(L \sim \lambda\) --- резонансная область. К её решению выделяют три
  основных подхода:
  \begin{itemize}
    \item метод Фурье (разделение переменных в волновом уравнении);
    \item метод запаздывающих потенциалов;
    \item метод векторного и скалярного интегралов Кирхгофа (хорошо описывает
    поведение волны за экраном с апертурой; в случае, если этот экран
    металлический, из-за токов возникают сложности).
  \end{itemize}
  \item \(L \gg \lambda\) --- квазиоптическая область. Основные методы решения
  -- геометрическая оптика (без дифракции); уравнение эйконала и геометрическая
  теория дифракции.
\end{itemize}

\section{Энергетические соотношения}
Итак, допустим, мы получили решение уравнений Максвелла, например, в виде
плоской волны:
\begin{align*}
  \vec{E} &= \vec{E_0} \exp\left[i(\omega t - \vec{\kappa}\vec{r}
  + \phi(t))\right].
\end{align*}
Возникает вопрос: как получить его экспериментальное подтверждение? Для этого
нужно измерить:
\begin{itemize}
  \item амплитуду \(\abs{\vec{E_0}}\);
  \item поляризацию (направление \(\vec{E_0}\);
  \item частоту \(\omega\);
  \item длину волны \(\lambda = \frac{2\pi}{\kappa}\);
  \item фазу \(\phi(t)\).
\end{itemize}
При этом наиболее точно мы можем измерить амплитуду; поляризацию --- примерно
в \(10^4\) раз грубее (методы эллипсометрии); частоту и длину волны --- примерно
с той же точностью, что и поляризацию (определяется имеющимися стандартами
частоты и длины); фазу измерять мы вообще не можем --- только разность фаз.

Так как амплитуда измеряется наиболее точно, для экспериментального
подтверждения уравнений Максвелла важно получить энергетические соотношения.

\begin{remark}[Об опыте Лебедева с пондеромоторной силой (передача импульса от
света)]
  Этот опыт был проделан более 100~лет назад, но при этом до сих пор ни
  в школьные, ни в студенческие лабы его не включают. Это печально,
  потому что опыт очень показательный и не требует сложного оборудования.
\end{remark}

\subsection{Теорема Умова --- Пойнтинга}
Запишем два уравнения Максвелла, для простоты опустив токи:
\begin{align*}
  \left[\vec{\nabla} \times \vec{H}\right] &= \pdiff{\vec{D}}{t} \\
  \left[\vec{\nabla} \times \vec{E}\right] &= -\pdiff{\vec{B}}{t}
\end{align*}
Домножив уравнения скалярно на \(\vec{E}\) и \(\vec{H}\) соответственно,
вычтем их друг из друга:
\begin{align*}
  \vec{E}\cdot\left[\vec{\nabla} \times \vec{H}\right] -
  \vec{H}\cdot\left[\vec{\nabla} \times \vec{E}\right] &= 
  \vec{E}\pdiff{\vec{D}}{t} - \vec{H}\pdiff{\vec{B}}{t}.
\end{align*}
После этого воспользуемся векторным тождеством
\begin{align*}
  \vec{E}\cdot\left[\vec{\nabla} \times \vec{H}\right]
  -
  \vec{H}\cdot\left[\vec{\nabla} \times \vec{E}\right]
  &= 
  -\div\left[\vec{E}\times\vec{H}\right].
\end{align*}
С другой стороны, воспользуемся тем, что
\begin{align*}
  \epsilon \pdiff{}{t}\vec{E}^2 = \epsilon \pdiff{E^2}{E}\pdiff{E}{t}
  = 2 E \pdiff{(\epsilon E)}{t},
\end{align*}
то есть,
\begin{align*}
  \vec{E}\pdiff{\vec{D}}{t} = \frac{1}{2}\epsilon\pdiff{E^2}{t}, 
\end{align*}
и, аналогично,
\begin{align*}
  \vec{H}\pdiff{\vec{B}}{t} = \frac{1}{2}\mu\pdiff{H^2}{t}.
\end{align*}
Пользуясь полученными равенствами, получаем
\begin{align*}
  \pdiff{}{t}\left[ \frac{\epsilon E^2 + \mu H^2}{2} \right]
  +\div\left[\vec{E}\times\vec{H}\right] = 0.
\end{align*}
Это закон сохранения в дифференциальной форме. Речь идёт о сохранении энергии,
отсюда плотность энергии
\begin{align*}
  w = 
  \frac{\epsilon E^2 + \mu H^2}{2},
\end{align*}
а плотность потока энергии (вектор Умова --- Пойнтинга) ---
\begin{align*}
  \vec{S} = \left[\vec{E} \times \vec{H}\right].
\end{align*}
Закон сохранения легко перевести из дифференциальной формы в интегральную,
проинтегрировав по некоторому замкнутому объёму, и, применив теорему Гаусса ---
Остроградского, перейти от интеграла дивергенции по объёму к интегралу по
поверхности:
\begin{align*}
  \pdiff{}{t}\iiint_{V} w dV = 
  \iiint_{V} \div\vec{S} dV = 
  \iint_{\partial V} \vec{S} \cdot d\vec{\sigma}.
\end{align*}
Поток энергии через единичную область представляет собой интенсивность:
\begin{align*}
  I = 
  \iint_{1\times 1} \left[\vec{E} \times \vec{H}\right] d\vec{\sigma} = 
  \iint_{1\times 1} \left[\vec{E} \times
  \frac{\sqrt{\mu_0\mu}}{\sqrt{\epsilon_0\epsilon}}\vec{E}\right] d\vec{\sigma} = \frac{\abs{E}^2}{Z},
\end{align*}
где мы воспользовались соотношением для плоских волн
\begin{align*}
  \sqrt{\epsilon_0\epsilon}\vec{E} = \sqrt{\mu_0\mu}\vec{H}
\end{align*}
и ввели импеданс
\begin{align*}
  Z = \frac{\sqrt{\epsilon_0\epsilon}}{\sqrt{\mu_0\mu}}.
\end{align*}
Таким образом, для плоской волны \(\vec{E} = \vec{E_0}\cos{\omega t}\),
\begin{align*}
  I = \frac{E_0^2}{Z} \cos^2\omega t.
\end{align*}
Автор конспекта предлагает обратить внимание на сходство с законом Джоуля ---
Ленца:
\begin{align*}
  P = \frac{U_0^2}{Z},
\end{align*}
причём электрическое поле измеряется в \(\text{В}/\text{м}\), а интенсивность ---
в~\(\text{Вт}/\text{м}^2\).

В реальном эксперименте в оптическом диапазоне детектор не может уследить за
множителем \(\cos\omega t\), поэтому наблюдается усреднённое значение
интенсивности:
\begin{align*}
  I = \frac{1}{\tau}\int_{t_0}^{t_0 + \tau}\frac{E_0^2}{Z} \cos^2\omega
  t = \frac{E_0^2}{2Z}.
\end{align*}
Именно этой формулой обычно и пользуются. В системе СГС импеданс вакуума равен~1.
  
