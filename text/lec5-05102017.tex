\section{Предельные случаи в электродинамике}
\begin{itemize}
  \item \(L \ll \lambda\) --- квазистатическая область. Считается
  канонизированной и полностью решённой.
  \item \(L \sim \lambda\) --- резонансная область. К её решению выделяют три
  основных подхода:
  \begin{itemize}
    \item метод Фурье (разделение переменных в волновом уравнении);
    \item метод запаздывающих потенциалов;
    \item метод векторного и скалярного интегралов Кирхгофа (хорошо описывает
    поведение волны за экраном с апертурой; в случае, если этот экран
    металлический, из-за токов возникают сложности).
  \end{itemize}
  \item \(L \gg \lambda\) --- квазиоптическая область. Основные методы решения
  -- геометрическая оптика (без дифракции); уравнение эйконала и геометрическая
  теория дифракции.
\end{itemize}

\section{Энергетические соотношения}
