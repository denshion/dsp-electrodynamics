\section{Предельные случаи в электродинамике}
\begin{itemize}
  \item \(L \ll \lambda\) --- квазистатическая область. Считается
  канонизированной и полностью решённой.
  \item \(L \sim \lambda\) --- резонансная область. К её решению выделяют три
  основных подхода:
  \begin{itemize}
    \item метод Фурье (разделение переменных в волновом уравнении);
    \item метод запаздывающих потенциалов;
    \item метод векторного и скалярного интегралов Кирхгофа (хорошо описывает
    поведение волны за экраном с апертурой; в случае, если этот экран
    металлический, из-за токов возникают сложности).
  \end{itemize}
  \item \(L \gg \lambda\) --- квазиоптическая область. Основные методы решения
  -- геометрическая оптика (без дифракции); уравнение эйконала и геометрическая
  теория дифракции.
\end{itemize}

\section{Энергетические соотношения}
Итак, допустим, мы получили решение уравнений Максвелла, например, в виде
плоской волны:
\begin{align*}
  \vec{E} &= \vec{E_0} \exp\left[i(\omega t - \vec{\kappa}\vec{r}
  + \phi(t))\right].
\end{align*}
Возникает вопрос: как получить его экспериментальное подтверждение? Для этого
нужно измерить:
\begin{itemize}
  \item амплитуду \(\abs{\vec{E_0}}\);
  \item поляризацию (направление \(\vec{E_0}\);
  \item частоту \(\omega\);
  \item длину волны \(\lambda = \frac{2\pi}{\kappa}\);
  \item фазу \(\phi(t)\).
\end{itemize}
При этом наиболее точно мы можем измерить амплитуду; поляризацию --- примерно
в \(10^4\) раз грубее (методы эллипсометрии); частоту и длину волны --- примерно
с той же точностью, что и поляризацию (определяется имеющимися стандартами
частоты и длины); фазу измерять мы вообще не можем --- только разность фаз.

Так как амплитуда измеряется наиболее точно, для экспериментального
подтверждения уравнений Максвелла важно получить энергетические соотношения.

\begin{remark}[Об опыте Лебедева с пондеромоторной силой (передача импульса от
света)]
  Этот опыт был проделан более 100~лет назад, но при этом до сих пор ни
  в школьные, ни в студенческие лабы его не включают. Это печально,
  потому что опыт очень показательный и не требует сложного оборудования.
\end{remark}

\subsection{Теорема Умова --- Пойнтинга}
Запишем два уравнения Максвелла, для простоты опустив токи:
\begin{align*}
  \left[\vec{\nabla} \times \vec{H}\right] &= \pdiff{\vec{D}}{t} \\
  \left[\vec{\nabla} \times \vec{E}\right] &= -\pdiff{\vec{B}}{t}
\end{align*}
Домножив уравнения скалярно на \(\vec{E}\) и \(\vec{H}\) соответственно,
вычтем их друг из друга:
\begin{align*}
  \vec{E}\cdot\left[\vec{\nabla} \times \vec{H}\right] -
  \vec{H}\cdot\left[\vec{\nabla} \times \vec{E}\right] &= 
  \vec{E}\pdiff{\vec{D}}{t} - \vec{H}\pdiff{\vec{B}}{t}.
\end{align*}
После этого воспользуемся векторным тождеством
\begin{align*}
  \vec{E}\cdot\left[\vec{\nabla} \times \vec{H}\right]
  -
  \vec{H}\cdot\left[\vec{\nabla} \times \vec{E}\right]
  &= 
  -\div\left[\vec{E}\times\vec{H}\right].
\end{align*}
С другой стороны, воспользуемся тем, что
\begin{align*}
  \epsilon \pdiff{}{t}\vec{E}^2 = \epsilon \pdiff{E^2}{E}\pdiff{E}{t}
  = 2 E \pdiff{(\epsilon E)}{t},
\end{align*}
то есть,
\begin{align*}
  \vec{E}\pdiff{\vec{D}}{t} = \frac{1}{2}\epsilon\pdiff{E^2}{t}, 
\end{align*}
и, аналогично,
\begin{align*}
  \vec{H}\pdiff{\vec{B}}{t} = \frac{1}{2}\mu\pdiff{H^2}{t}.
\end{align*}
Пользуясь полученными равенствами, получаем
\begin{align*}
  \pdiff{}{t}\left[ \frac{\epsilon E^2 + \mu H^2}{2} \right]
  +\div\left[\vec{E}\times\vec{H}\right] = 0.
\end{align*}
Это закон сохранения в дифференциальной форме. Речь идёт о сохранении энергии,
отсюда плотность энергии
\begin{align*}
  w = 
  \frac{\epsilon E^2 + \mu H^2}{2},
\end{align*}
а плотность потока энергии (вектор Умова --- Пойнтинга) ---
\begin{align*}
  \vec{S} = \left[\vec{E} \times \vec{H}\right].
\end{align*}
Закон сохранения легко перевести из дифференциальной формы в интегральную,
проинтегрировав по некоторому замкнутому объёму, и, применив теорему Гаусса ---
Остроградского, перейти от интеграла дивергенции по объёму к интегралу по
поверхности:
\begin{align*}
  \pdiff{}{t}\iiint_{V} w dV = 
  \iiint_{V} \div\vec{S} dV = 
  \iint_{\partial V} \vec{S} \cdot d\vec{\sigma}.
\end{align*}
Поток энергии через единичную область представляет собой интенсивность:
\begin{align*}
  I = 
  \iint_{1\times 1} \left[\vec{E} \times \vec{H}\right] d\vec{\sigma} = 
  \iint_{1\times 1} \left[\vec{E} \times
  \frac{\sqrt{\mu_0\mu}}{\sqrt{\epsilon_0\epsilon}}\vec{E}\right] d\vec{\sigma} = \frac{\abs{E}^2}{Z},
\end{align*}
где мы воспользовались соотношением для плоских волн
\begin{align*}
  \sqrt{\epsilon_0\epsilon}\vec{E} = \sqrt{\mu_0\mu}\vec{H}
\end{align*}
и ввели импеданс
\begin{align*}
  Z = \frac{\sqrt{\epsilon_0\epsilon}}{\sqrt{\mu_0\mu}}.
\end{align*}
Таким образом, для плоской волны \(\vec{E} = \vec{E_0}\cos{\omega t}\),
\begin{align*}
  I = \frac{E_0^2}{Z} \cos^2\omega t.
\end{align*}
Автор конспекта предлагает обратить внимание на сходство с законом Джоуля ---
Ленца:
\begin{align*}
  P = \frac{U_0^2}{Z},
\end{align*}
причём электрическое поле измеряется в \(\text{В}/\text{м}\), а интенсивность ---
в~\(\text{Вт}/\text{м}^2\).

В реальном эксперименте в оптическом диапазоне детектор не может уследить за
множителем \(\cos\omega t\), поэтому наблюдается усреднённое значение
интенсивности:
\begin{align*}
  I = \frac{1}{\tau}\int_{t_0}^{t_0 + \tau}\frac{E_0^2}{Z} \cos^2\omega
  t = \frac{E_0^2}{2Z}.
\end{align*}
Именно этой формулой обычно и пользуются. В системе СГС импеданс вакуума равен~1.


\section{Распространение электромагнитных волн в различных средах}
В данном параграфе будем рассматривать два вида сред:
\begin{itemize}
  \item проводящая среда с диэлектрической и магнитной проницаемостью;
  \item идеальный металл (\(\sigma \to \infty\)).
\end{itemize}

Наша задача --- дать объяснение следующим экспериментальным фактам:
\begin{enumerate}
  \item закон Бугера: в среде волна затухает по закону
  \begin{align*}
    I(z) = I_0\exp(-2\alpha z);
  \end{align*}
  \item от идеального металла свет отражается практически полностью;
  \item при отражении от металла свет меняет поляризацию.
\end{enumerate}

Для этого будем рассматривать модель комплексной диэлектрической проницаемости
\begin{align*}
  \epsilon = \epsilon' - i\epsilon'',
\end{align*}
где \(\epsilon''\) может быть связан как с поляризацией, так и с проводимостью:
в этом случае \(\epsilon'' = \frac{\sigma}{\omega}\).

\subsection{Закон Бугера}
Получим для среды с комплексной диэлектрической проницаемостью коэффициент
затухания \(\alpha\) и вещественную часть \(\beta\) волнового вектора.
В этом месте в оптике и в радиофизике обозначения различаются (для сравнения,
величины, помеченные <<крышкой>> --- комплексные, остальные --- вещественные):
\begin{align*}
  &\text{оптика}&
  &\text{радиофизика}\\
  \hat{\kappa} &= \frac{\omega}{\hat{v}_{\phi}}, &
  \beta &= \frac{\omega}{v_{\phi}}, \\
  \hat{\kappa} &= \beta - i\alpha, &
  \hat{\gamma} &= \alpha + i\beta. \\
\end{align*}
В радиофизике \emph{постоянную распространения} \(\gamma\) вводят таким образом,
чтобы выполнялось \(P_{\text{out}} = P_{\text{in}}\exp(\gamma l)\). По-видимому,
такое отступление связано с тем, что в радиофизике не рассматривают комплексную
диэлектрическую проницаемость, вводя вместо этого проводимость и поляризацию
отдельно.

Мы будем пользоваться оптической системой обозначений, так как в нашем
рассмотрении <<фазовая скорость комплексная>>.

Итак, нас интересует нахождение \(\alpha\) и \(\beta\) из
\begin{align*}
  \kappa = \frac{\omega}{c}\sqrt{\epsilon' - i\epsilon''}\sqrt{\mu} = \beta - i\alpha.
\end{align*}
Для этого воспользуемся выражением для квадратного корня комплексного числа:
\begin{align*}
  \sqrt{z} = \sqrt{\frac{\abs{z}}{2} + \frac{\Re{z}}{2}} \pm
  i\sqrt{\frac{\abs{z}}{2} - \frac{\Re{z}}{2}}.
\end{align*}
Выразив таким образом \(\sqrt{\epsilon' - i\epsilon''}\), можем разбить
уравнение на вещественную и мнимую части:
\begin{align*}
  \beta &= \frac{\omega}{c}\sqrt{\mu}\sqrt{\frac{\sqrt{(\epsilon')^2 + (\epsilon'')^2}}{2} + \frac{\epsilon'}{2}}, &
  \alpha &= \frac{\omega}{c}\sqrt{\mu}\sqrt{\frac{\sqrt{(\epsilon')^2 + (\epsilon'')^2}}{2} - \frac{\epsilon'}{2}}. \\
\end{align*}
Вынося из-под корня \(\epsilon'/2\), получаем:
\begin{align*}
  \alpha &= \frac{\omega}{c}\sqrt{\frac{\epsilon'\mu}{2}\left[\sqrt{1
  + \left(\frac{\epsilon''}{\epsilon'}\right)} - 1\right]}, &
  \beta &= \frac{\omega}{c}\sqrt{\frac{\epsilon'\mu}{2}\left[\sqrt{1
  + \left(\frac{\epsilon''}{\epsilon'}\right)} + 1\right]}.
\end{align*}
В случае металла можно перейти к проводимости: \(\epsilon''
= \sigma/\omega\epsilon_0\).
Тогда 
\begin{align*}
  \alpha &= \frac{\omega}{c}\sqrt{\frac{\epsilon'\mu}{2}\left[\sqrt{1
  + \left(\frac{\sigma}{\omega\epsilon_0\epsilon'}\right)} - 1\right]}, &
  \beta &= \frac{\omega}{c}\sqrt{\frac{\epsilon'\mu}{2}\left[\sqrt{1
  + \left(\frac{\sigma}{\omega\epsilon_0\epsilon'}\right)} + 1\right]}.
\end{align*}

\subsection{Скин-эффект}
В случае идеального металла --- когда \(\sigma \gg \omega\epsilon_0\epsilon'\), ---
формулы упрощаются. Действительно, пользуясь соотношением
\begin{align*}
  \lim\limits_{x \to \infty} \frac{\sqrt{1 + x^2} \pm 1}{x} = 1,
\end{align*}
можем упростить формулы для \(\alpha\) и \(\beta\) до
\begin{align*}
  \alpha = \beta
  = \frac{\omega}{c}\sqrt{\frac{\epsilon'\mu}{2}\frac{\sigma}{\omega\epsilon_0\epsilon'}}
  = \frac{1}{c}\sqrt{\frac{\omega\sigma\mu}{2\epsilon_0}}.
\end{align*}

Отсюда можно получить глубину скин-слоя: это глубина, на которой, по закону
Бугера, поле затухает в \(e\) раз --- другими словами, \(\alpha d_s = 1\).
Тогда
\begin{align*}
  d_s = \alpha^{-1} = c\sqrt{\frac{2\epsilon_0}{\omega\sigma\mu}}.
\end{align*}

