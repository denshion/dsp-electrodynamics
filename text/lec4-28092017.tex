\section{Временная дисперсия на языке плоских волн. Формула Релея}
Как было показано в предыдущих параграфах, фазовая скорость плоской волны
равна
\begin{align*}
  v_{\phi} = \frac{\omega}{\kappa} = \frac{c}{n}
  = \frac{c}{\sqrt{\epsilon}{\mu}}.
\end{align*}

Это скорость движения фронта плоской монохроматической волны. Что же произойдёт
если в среду запустить волновой пакет --- набор плоских волн с различными
частотами? Будет наблюдаться \emph{временная дисперсия} --- расплывание
волнового пакета.

Для анализа прохождения волнового пакета через среду вводится понятие
\emph{групповой скорости}:
\begin{align*}
  v_{\text{гр.}} \equiv \left(\pdiff{\omega}{\kappa}\right)_{\kappa = \kappa_0},
\end{align*}
где волновой вектор \(\kappa_0\) соответствует центру спектра волнового пакета
(?).

Легко формально показать, что
\begin{align*}
  v_{\text{гр.}} = v_{\phi} + \kappa \pdiff{v_{\phi}}{\kappa}
\end{align*}
Эту формулу и называют формулой Релея. Но можно пойти дальше и перевести всё на
язык оптики, в котором работают с показателем преломления и длиной волны. Тогда
формула будет выглядеть следующим образом:
\begin{align*}
  v_{\text{гр.}} &= v_{\phi} - \lambda \pdiff{v_{\phi}}{\lambda} \\
  \lambda\pdiff{v_{\phi}}{\lambda} &= -\frac{c}{n^2}\pdiff{n}{\lambda}.
\end{align*}

Здесь мы получаем три случая:
\begin{itemize}
  \item \(v_{\text{гр.}} = v_{\phi}\), или же \(\pdiff{n}{\lambda} = 0\) ---
    отсутствие дисперсии;
  \item \(v_{\text{гр.}} < v_{\phi}\), или же \(\pdiff{n}{\lambda} < 0\) ---
    нормальная дисперсия;
  \item \(v_{\text{гр.}} > v_{\phi}\), или же \(\pdiff{n}{\lambda} > 0\) ---
    аномальная дисперсия.
\end{itemize}

// привести картинку \(\varepsilon\) с аномальной дисперсией.


\section{Пространственные частоты. Пространственная фильтрация. Параксиальное
приближение}
Из выражения
\begin{align*}
  \frac{c}{n} = \frac{\omega}{\kappa}
\end{align*}
легко получить
\begin{align*}
  \frac{\kappa^2_x + \kappa^2_y + \kappa^2_z}{\epsilon\mu}
  = \frac{\omega^2}{c^2}.
\end{align*}
В этом выражении уже видна некоторая аналогия между волновыми векторами и частотами.
Это побуждает рассматривать волновые векторы как <<пространственные частоты>>.

В предыдущем параграфе рассматривался волновой пакет из плоских волн; в общем же
случае у волн в волновом пакете различается не только модуль волнового вектора
(что соответствует различным частотам), но и его направление. С этим связано
понятие \emph{пространственной дисперсии}: пусть на рассеивающую среду падает
плоская волна. В результате рассеяния появляются вторичные волны,
распространяющиеся в направлениях, отличных от изначального. То есть, у них
изменились направления волнового вектора. Такое изменение и называется
пространственной дисперсией.

С этим же тесно связано и понятие пространственной фильтрации: если приёмник
с некоторой конечной апертурой установить слишком далеко от рассеивающей среды,
то слишком сильно рассеянные волны просто не попадут в эту апертуру --- они будут
отфильтрованы. Стоит заметить, что если мы не хотим потерять всей информации
о волновом пакете (то есть, и частотного, и пространственного спектра), то
пространственная фильтрация нежелательна, так как ведёт к потере части
рассеянных волн.

\subsection{Математический аппарат пространственной дисперсии}
Выберем ось \(z\) вдоль направления распространения волнового пакета как целого.
К полю можно применить преобразование Фурье по компонентам волнового вектора,
перпендикулярным направлению распространения:
\begin{align*}
  \vec{E}(\kappa_x, \kappa_y, z=0) &= \frac{1}{4\pi^2}\iint_{-\infty}^{+\infty}
  \vec{E}(x, y, z=0) \exp\left[i(-\kappa_x x - \kappa_y y)\right] dxdy \\
  \vec{E}(x, y, z=0) &= \iint_{-\infty}^{+\infty} \vec{E}(\kappa_x, \kappa_y, z=0)
  \exp\left[i(\kappa_x x + \kappa_y y)\right] d\kappa_xd\kappa_y. 
\end{align*}
Если теперь в таком виде подставить электрическое поле в уравнение Гельмгольца
\begin{align*}
  \nabla^2\vec{E} + \kappa_z^2\vec{E} = 0,
\end{align*}
то получим решение в виде
\begin{align*}
  \vec{E}(\kappa_x, \kappa_y, z) = \vec{E}(\kappa_x, \kappa_y, z=0)\exp(\pm
  i\kappa_z z).
\end{align*}
Переходя к Фурье-прообразу, получаем:
\begin{align*}
  \vec{E}(x, y, z) = \iint_{-\infty}^{+\infty} \vec{E}(\kappa_x, \kappa_y, z=0)
  \exp\left[i(\kappa_xx + \kappa_yy + \kappa_zz)\right] d\kappa_xd\kappa_y.
\end{align*}
Таким образом, мы получили спектр по \(\kappa_x\) и \(\kappa_y\).

\subsection{Замечание о разрешении изображений}
В классической оптике разрешение ограничивается критерием Релея --- Джинса,
т.\,е., не точнее \(\frac{\lambda}{2}\). Однако если располагать информацией о:
\begin{itemize}
  \item частотном спектре \(\vec{E}(\omega)\);
  \item пространственном спектре \(\vec{E}(\kappa_x, \kappa_y)\);
  \item эванесцентных модах,
\end{itemize}
--- то этот предел разрешения можно преодолеть (см., например, работы Белова).

\subsection{Параксиальное приближение}
Рассмотрим поле в дальней зоне: \(r \gg \lambda\). Квазиплоское в гауссовом
приближении поле можно разложить по пространственным частотам. При этом
\begin{align*}
  \kappa^2 = \kappa^2_x + \kappa^2_y + \kappa^2_z,
\end{align*}
откуда
\begin{align*}
  \kappa_z = \sqrt{\kappa^2 - \kappa^2_x - \kappa^2_y} \simeq \kappa
  - \frac{\kappa_x^2 + \kappa_y^2}{2\kappa}.
\end{align*}
Применённое здесь разложение в ряд и составляет суть параксиального приближения.
В параксиальном приближении получается критерий Релея --- Джинса.
\begin{remark}
  Дифракционный угол одномодового лазера --- \(\phi \simeq 1.22\lambda/D\).
\end{remark}

\section{Когерентность}
Поле любого реального источника необходимо представлять в виде
\begin{align*}
  \vec{E} = \vec{E_0} \exp\left[i(\omega t - \vec{\kappa}\vec{r}
  + \phi(t))\right],
\end{align*}
где \(\phi(t)\) --- случайная фаза. 
\begin{remark}[Ремарка от меня]
  Если хочется порассуждать о когерентности в радиотехнике, то там выглядит
  более разумным рассуждать не о случайной фазе, а о случайном изменении частоты
  --- например, из-за теплового расширения резонатора. Но это всё лирика.
\end{remark}
На тему этой случайной фазы можно рассуждать много и долго\footnote{Например, эргодическая
гипотеза, которой все пользуются --- никакого обоснования, кроме как ,,ну,
наверное, это так'' под собой не имеет. Кто жаждет подробностей --- они есть
в книге <<Когерентность света>> Яна Перины.}, но мы здесь рассмотрим лишь одно
предположение: что случайное изменение фазы --- это \emph{стационарный случайный
процесс}. Стационарность здесь означает однородность во времени всех усредняемых
параметров:
\begin{align*}
  \forall t_1, t_2 \quad \frac{1}{\tau}\int_{t_1}^{t_1 + \tau} f(t)dt = 
  \frac{1}{\tau}\int_{t_2}^{t_2 + \tau} f(t)dt
\end{align*}

Итак, рассмотрим опыт Юнга --- двухлучевую интерференцию. Поля от двух
точечных источников (или узких щелей) складываются и, возможно, образуют на
экране интерференционную картину. Запишем для этого случая квадрат поля, считая
его, для простоты, скалярным:
\begin{align*}
  \abs{E_1 + E_2}^2 = \abs{E_1}^2 + \abs{E_2}^2 + 2\abs{E_1E_2}
\end{align*}
При этом интенсивность, регистрируемая
квадратичным детектором, будет выражаться следующим образом:
\begin{align*}
  \frac{1}{\tau}\int_{t_0}^{t_0 + \tau} \abs{E_1 + E_2}^2 dt
  = \frac{1}{\tau}\int_{t_0}^{t_0 + \tau} \abs{E_1}^2dt 
  + \frac{1}{\tau}\int_{t_0}^{t_0 + \tau} \abs{E_2}^2dt 
  + 2\frac{1}{\tau}\int_{t_0}^{t_0 + \tau} \abs{E_1E_2}dt.
\end{align*}
Введя измеряемые интенсивности
\begin{align*}
  I_i = \frac{1}{\tau}\int_{t_0}^{t_0 + \tau} \abs{E_i}^2 dt,
\end{align*}
можем переписать это в виде
\begin{align*}
  I_{1+2} = I_1 + I_2 + 2\sqrt{I_1I_2}\abs{\gamma_{21}(\tau)},
\end{align*}
где \(\gamma_{21}(\tau)\) (далее --- просто \(\gamma(\tau)\)) --- степень
когерентности, определённая, как
\begin{align*}
  \gamma(\tau) = \cfrac{
  \left[\frac{1}{\tau}\int_{t_0}^{t_0 + \tau} \abs{E_1E_2}dt \right]
  }{
  \sqrt{
    \left[\frac{1}{\tau}\int_{t_0}^{t_0 + \tau} \abs{E_1}^2dt\right]
    \left[\frac{1}{\tau}\int_{t_0}^{t_0 + \tau} \abs{E_2}^2dt\right] 
  }
  }.
\end{align*}
Как нетрудно заметить, \(0 \le \abs{\gamma(\tau)} \le 1\), причём
\(\gamma(\tau) = 0\) соответствует полностью некогерентному свету (складываются
интенсивности), а \(\gamma(\tau)\) --- полностью когерентному (складываются
амплитуды).

Экспериментальным способом определения когерентности является опыт Юнга.
// расписать это здесь как следует

// Нужно упомянуть теоремы Винера --- Хинчина и ван Циттера --- Цернике, но я не
помню, с какой целью

// Рассказать кулстори про опподизацию и геометрическую дифракцию




