\section{Решение уравнений Максвелла}
REFINE!!!
\subsection{Принцип перестановочной двойственности}
Запишем уравнения Максвелла, разделив предварительно токи и заряды на внутренние
и сторонние:
\begin{align*}
	\div{\vec{D}} &= \rho + \rho_{\text{ст.}} &
	\div{\vec{D}} &= 4\pi(\rho + \rho_{\text{ст.}}) \\
	\div{\vec{B}} &= 0 &
	\div{\vec{B}} &= 0 \\
	\rot{\vec{E}} &= -\pdiff{\vec{B}}{t} &
	\rot{\vec{E}} &= -\frac{1}{c}\pdiff{\vec{B}}{t} \\
	\rot{\vec{H}} &= \vec{j}(\vec{r}) + \vec{j_{\text{ст.}}}(\vec{r}) + \pdiff{\vec{D}}{t} &
	\rot{\vec{H}} &= \frac{4\pi}{c}\left(\vec{j}(\vec{r}) + \vec{j_{\text{ст.}}}(\vec{r})\right) + \frac{1}{c}\pdiff{\vec{D}}{t}
\end{align*}

Принцип перестановочной двойственности:
можно выполнить замену:
\begin{align*}
  \vec{E} &\to \vec{H}, &
  \sigma_{\text{эл.}} &\to \sigma_{\text{магн.}}, &
  \epsilon_0\epsilon &\to -\mu_0\mu, &
  \vec{j_{\text{эл.}}} &\to \vec{j_{\text{магн.}}}.
\end{align*}
Тогда получим уравнения для магнитных зарядов и токов.

\subsection{Подходы к решению уравнений Максвелла}
<<В прямую>> уравнения Максвелла не решаются. Существует четыре способа
сведения их к более удобной для решения форме:
\begin{enumerate}
  \item волновое уравнение для электрического и магнитного полей;
  \item волновое уравнение для скалярного и векторного потенциалов;
  \item волновое уравнение для электрического и магнитного векторов Герца;
  \item уравнение первого порядка для вектора Римана --- Зилберштейна.
\end{enumerate}
Для получения квантовой теории квантование удобно применять к последним двум
представлениям.

\subsection{Волновое уравнение для полей}
Из уравнений Максвелла легко получить волновое уравнение
\begin{align*}
  \nabla^2\vec{E} - \frac{\epsilon_0\epsilon\mu_0\mu}{c^2} \pdiff[2]{\vec{E}}{t} = 0.
\end{align*}
Простейшим решением данного уравнения является плоская волна:
\begin{align*}
  \vec{E} = \vec{E_0}\exp\left[i(\omega t - \vec{\kappa}\vec{r})\right].
\end{align*}
Фазовая скорость волны \(v_{\phi} = \omega/\kappa\):
\begin{align*}
  \frac{\epsilon_0\epsilon\mu_0\mu}{c^2} = \frac{1}{v_{\phi}^2}, 
\end{align*}
следовательно,
\begin{align*}
  v_{\phi} = \frac{c}{\sqrt{\epsilon_0\mu_0\epsilon\mu}} = \frac{c}{n}.
\end{align*}
Фаза постоянна в пространстве, когда \(\vec{\kappa}\vec{r} = \const\), что
являет собой уравнение плоскости. Поэтому такая волна --- плоская.

Другим решением является сферическая волна:
\begin{align*}
  \vec{E}(\vec{r}, t) = \frac{\vec{E}}{\abs{\vec{r}}} \exp\left[i(\omega
  t - \vec{\kappa}\vec{r})\right].
\end{align*}

// Получить закон сохранения заряда и волновые уравнения с \(\rho\)
и \(\vec{j} \neq 0\).

\subsection{Векторный и скалярный потенциалы}
Скалярный потенциал \(\phi\) и векторный потенциал \(\vec{A}\) вводятся
следующим образом:
\begin{align*}
  \vec{E} &= -\grad{\phi} - \pdiff{\vec{A}}{t}, \\
  \vec{B} &= \rot{\vec{A}}.
\end{align*}

Для них можно получить волновые уравнения
\begin{align*}
  \nabla^2 A - \frac{\epsilon\mu}{c^2} \pdiff[2]{\vec{A}}{t} &= -\mu_0\mu
  \vec{j_{\text{ст.}}}, \\
  \nabla^2\phi - \frac{\epsilon\mu}{c^2} \pdiff[2]{\phi}{t} &=
  -\frac{\rho_{\text{ст.}}}{\epsilon_0\epsilon}.
\end{align*}
// калибровка?

//Доказать калибровочную инвариантность.

\subsection{Волновые уравнения для векторов Герца}
Вводят два вектора Герца: электрический и магнитный.
// вывести это всё и написать здесь.
\subsection{Вектор Римана --- Зильберштейна}
// привести всё это здесь.

