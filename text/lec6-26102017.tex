\section{Модель Друде --- Лоренца --- Зоммерфельда}
В металлах диэлектрическая проницаемость описывается формулой Дебая:
\begin{equation}
  \varepsilon^{*} = \varepsilon_{\infty} 
-
\underbrace{
  \frac{\varepsilon_0 - \varepsilon_{\infty}}{1 + i\omega \tau}
}_{\text{резонансное поглощение}}
  -
\underbrace{
i\frac{\sigma}{\omega\varepsilon_0}
}_{\text{экранировка}}
\end{equation}

Для примера рассмотрим параметры трёх лучших металлов: Ag, Au, Cu.
В ИК-диапазоне у них наблюдается показатель преломления \(n < 1\).
Как это объяснить?

При построении теории оптики хороших металлов, следует иметь ввиду, что для
них \(\sigma \to \infty\) и \(R \to 1\). Для сравнения:
\(\sigma_{\text{бескислородн. Cu}} \sim 10^7\ldots10^8\).

\subsection{Отклик идеального газа электронов}
Будем рассматривать электронный газ в металле как идеальный.
Для построения модели Лоренца --- Друде --- Зоммерфельда проделываются следующие
шаги:
\begin{enumerate}
\item
Записываются силы, действующие на электрон --- сила трения \(m\gamma\dot{r}\)
и взаимодействие с электрическим полем волны \(eEe^{i\omega t}\).
\item
Из них получается уравнение 
\begin{equation}
  \ddot{r} + \gamma\dot{r} = E\frac{e}{m} e^{i\omega t}.
\end{equation}
Решение этого уравнения ---
\begin{equation}
  r(t) = -Ee^{i\omega t} \frac{e}{m} \frac{1}{\omega^2 - i\gamma\omega}.
\end{equation}
\item
Одиночный электрон обладает дипольным моментом \(e r(t)\); все электроны
в единице объёма вещества ---
\begin{equation}
  P = \frac{1}{V} \sum_i p_i = ner(t),
\end{equation}
где \(n\) --- концентрация электронов.
\item
Так как \(\vec{D} = \varepsilon_0 \vec{E} + \vec{P}\), 
\begin{equation}
  \varepsilon(\omega) = \frac{De^{i\omega t}}{Ee^{i\omega t}} = \varepsilon_0 + \frac{ner(t)}{Ee^{i\omega t}}
  =
  \varepsilon_0 - \frac{ne^2}{m}\frac{1}{\omega^2 - i\gamma\omega}.
\end{equation}
\end{enumerate}
Таким образом, получаем отклик в виде 
\begin{equation}
  \varepsilon(\omega) = \varepsilon_0 \left[1 - \frac{\omega_p^2}{\omega^2
  - i\gamma\omega}\right],
\end{equation}
где \(\omega^2_p = e^2n/m\varepsilon_0\) называется \emph{плазменной частотой}.

Для хороших металлов \(\gamma \ll \omega\), поэтому диэлектрическая
проницаемость для них имеет вид
\begin{equation}
  \varepsilon(\omega) = \varepsilon_0\left[1
  - \frac{\omega_p^2}{\omega^2}\right].
\end{equation}
При этом, как легко видеть, в области \(\omega > \omega_p\) наблюдается \(n
< 1\).

// Может, в начале про \(n < 1\) речь шла про ультрафиолет, а не ИК?

В качестве характеристики потерь можно ввести \emph{тангенс угла потерь}, равный
\begin{align*}
  \tg\varphi = \frac{\varepsilon''}{\varepsilon'}.
\end{align*}

Подробнее почитать можно в: Климове (наноплазмоника); Сарычев, Шалаев ---
<<Электродинамика метаматериалов>>.

\subsection{Смысл плазменной частоты}
Уравнению Максвелла \(\div \vec{D} = \varepsilon(\omega)\div\vec{E} = 0\)
можно удовлетворить двумя способами:
\begin{itemize}
  \item \(\div\vec{E} = 0\) --- поперечная волна;
  \item \(\varepsilon(\omega) = 0\) --- продольная волна.
\end{itemize}

Разберём эти два случая по-отдельности.
\paragraph{Поперечная волна}
Для поперечной волны уравнение Гельмгольца имеет вид
\begin{equation}
  \underbrace{
  -\vec{\kappa}
  (\vec{\kappa}\cdot\vec{E})
  }_{= 0}
  + \kappa^2 \vec{E}(\vec{\kappa}, \omega) = \varepsilon(\vec{\kappa}, \omega)
  \frac{\omega^2}{c^2} \vec{E}(\vec{\kappa}, \omega).
\end{equation}

Отсюда получаем дисперсионное соотношение:
\begin{equation}
  \omega \varepsilon(\vec{\kappa}, \omega) = c^2 \kappa^2.
\end{equation}

Если взять \(\varepsilon(\omega) = \varepsilon_0(1 - \omega_p^2 / \omega^2)\),
то получается дисперсионное соотношение поперечного поляритона:
\begin{equation}
  \omega^2 = \omega_p^2 + c^2\kappa^2.
\end{equation}

\paragraph{Продольная волна}
В случае продольной волны (\(\varepsilon(\omega, \vec{\kappa}) = 0\)),
пространственную дисперсию исключать нельзя. 
