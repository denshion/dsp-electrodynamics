\subsection{Снова о методах}
Итак, если не рассматривать вектор Римана --- Зильберштейна, электрическое и
магнитное поля можно получить из уравнений Максвелла, преобразовав их к волновому уравнению.
Его можно записать для:
\begin{itemize}
	\item \(\vec{E}, \vec{H}\);
	\item \(\phi, \vec{A}\);
	\item \(\Gamma^{\text{эл.}}, \Gamma^{\text{магн.}}\).
\end{itemize}
Подход через векторы Герца привлекателен тем, что сразу даёт разделение на TE- и TM-поляризацию.
Это и есть общий матфизический подход к уравнениям Максвелла; всё остальное --- уже комбинаторика.

\subsection{Символический метод}
Если с самого начала искать решение в виде плоских монохроматических волн, или же, если угодно, подвергнуть уравнения
Максвелла преобразованию Фурье, то это приведёт к следующей замене, имеющей громкое название <<\emph{символический метод}>>:
\begin{align*}
	\pdiff{}{t} &\to i\omega &
	\nabla &\to i\vec\kappa.
\end{align*}
%Волновое уравнение при этом преобразуется в уравнение Гельмгольца:
%Из второго уравнения при этом л
Уравнение Гельмгольца (как возникает?)
\begin{align*}
	\nabla^2\vec{E} + \kappa^2\vec{E} = 0.
\end{align*}

\subsection{Плоские и монохроматические волны: применимость приближения}
В предыдущем параграфе мы решили искать решение уравнение Максвелла в виде
плоских монохроматических волн. Тем не менее, очевидно, что ни идеально плоских,
ни идеально монохроматических волн в природе не существует и существовать не
может, так как такая волна была бы бесконечной в пространстве либо во времени
соответственно. Поэтому говорят о квазиплоских и квазимонохроматических волнах.
Для начала, дадим определение этим понятиям.

\emph{Квазимонохроматическое излучение} --- излучение, ширина спектра которого
много меньше частоты (\(\Delta\omega/\omega \ll 1\)).

Без применения специальных мер невозможно получить спектральную ширину линии
уже, чем естественная ширина, так как последняя является характеристикой атома,
а источники излучения из этих самых атомов состоят.

// получить и привести выражение для естественной ширины линии

Ширина линии одномодового лазера may be as small as \(~1~\text{МГц}\).
\begin{remark}
При этом методами нелинейной спектроскопии можно достигнуть на два порядка
меньших значений.
\end{remark}
Факторы, влияющие на уширение линии: столкновительный, допплеровский...

//здесь надо привести таблицу ширин линий источников: лазера, ВЧ-диода,
клистрона...

Теперь обсудим пространственные характеристики. Реальное излучение всегда
испускается источником конечного размера, поэтому его волновой фронт всегда
изогнут. Однако на большом расстоянии от источника волновые фронты имеют столь
большой радиус кривизны, что являются почти плоскими. Осталось пояснить, что
значит <<почти>>. Разумно сравнивать прогиб волнового фронта с длиной волны.
Таким образом, \emph{квазиплоская волна} --- это волна, прогиб волнового фронта
которой меньше или порядка длины волны. Стоит заметить, что прогиб волнового
фронта зависит от площади, на которой рассматривается излучение, поэтому 
квазиплоское в некоторой площади излучение может перестать быть квазиплоским,
если его рассматривать в более широкую апертуру.

Стоит рассмотреть и другой предельный случай --- сферическую волну. Волна
считается сферической, если размер источника меньше или порядка длины волны.
// чё-то тут нечисто

\section{Мнимая часть диэлектрической проницаемости. Модель Дебая}
\begin{align*}
  \epsilon = \epsilon' - i\epsilon''
\end{align*}
\begin{align*}
  \epsilon = \epsilon_{\infty} + \frac{\epsilon_{s}
  - \epsilon_{\infty}}{1 + i\omega\tau}
\end{align*}
Данная формула является чисто феноменологической, и не включает в себя механизма
поляризации.

Из комплексной диэлектрической проницаемости радиофизики рассматривают тангенс
угла потерь:
\begin{align*}
  \tg{\phi} = \frac{\epsilon'}{\epsilon''},
\end{align*}
а оптики -- коэффициенты преломления и экстинкции.

