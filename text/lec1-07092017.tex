%\section*{Введение}
%\section*{Литература}

\section{Уравнения Максвелла.}
В основе электродинамики лежат уравнения Максвелла. В дифференциальной форме они записываются следующим образом:
\begin{align*}
	\div{\vec{D}} &= \rho &
	\div{\vec{D}} &= 4\pi\rho \\
	\div{\vec{B}} &= 0 &
	\div{\vec{B}} &= 0 \\
	\rot{\vec{E}} &= -\pdiff{\vec{B}}{t} &
	\rot{\vec{E}} &= -\frac{1}{c}\pdiff{\vec{B}}{t} \\
	\rot{\vec{H}} &= \vec{j}(\vec{r}) + \pdiff{\vec{D}}{t} &
	\rot{\vec{H}} &= \frac{4\pi}{c}\vec{j}(\vec{r}) + \frac{1}{c}\pdiff{\vec{D}}{t}
\end{align*}

Интегральная форма этих же уравнений:
\begin{align*}
	\oint_{s} \vec{D}\cdot d\vec{\sigma} &= Q &
	\oint_{s} \vec{D} \cdot d\vec{\sigma} &= 4\pi Q \\
	\oint_{s} \vec{B} \cdot d\vec{\sigma} &= 0 &
	\oint_{s} \vec{B} \cdot d\vec{\sigma} &= 0 \\
	\oint_{l} \vec{E} \cdot d\vec{l} &= -\fdiff{}{t}\int_s \vec{B}\cdot d\vec{\sigma} &
	\oint_{l} \vec{E} \cdot d\vec{l} &= -\frac{1}{c}\fdiff{}{t}\int_s \vec{B}\cdot d\vec{\sigma} \\
	\oint_{l} \vec{H} \cdot d\vec{l} &= I + \fdiff{}{t}\int_s \vec{D} \cdot d\vec{\sigma} &
	\oint_{l} \vec{H} \cdot d\vec{l} &= \frac{4\pi}{c}I + \frac{1}{c}\fdiff{}{t}\int_s \vec{D} \cdot d\vec{\sigma}
\end{align*}

Дифференциальная форма уравнений получается из интегральной путём применения теорем Гаусса --- Остроградского и Стокса.

Стоит отметить, что уравнения Максвелла описывают лишь распространение волн, но не их генерацию.

\subsection{Материальные уравнения}
Для того, чтобы решить задачу электродинамики --- нахождение полей \(\vec{E}(\vec{r}, t)\), \(\vec{H}(\vec{r}, t)\), \(\vec{D}(\vec{r}, t)\) и \(\vec{B}(\vec{r}, t)\), эту систему из четырёх уравнений необходимо дополнить также материальными уравнениями, например,
\begin{align*}
	\vec{D} &= \epsilon_0\epsilon\vec{E} &
	\vec{D} &= \epsilon\vec{E} \\
	\vec{B} &= \mu_0\mu\vec{H} &
	\vec{B} &= \mu\vec{H} \\
	\vec{j} &= \sigma\vec{E} &
	\vec{j} &= \sigma\vec{E}.
\end{align*}
Также иногда требуется привлечь силу взаимодействия зарядов:
\begin{align*}
	\vec{F} &= q\left(\vec{E} + [\vec{v} \times \vec{B}]\right) & 
	\vec{F} &= q\left(\vec{E} + \frac{1}{c}[\vec{v} \times \vec{B}]\right).
\end{align*}

Сделаем несколько замечаний о материальных уравнениях. Что вообще микроскопически представляет из себя
диэлектрическая проницаемость \(\epsilon\)? Она описывает поляризацию --- переориентацию или изменение величины
 дипольных моментов, присутствующих в среде. Поля этих диполей вносят в
электрическое поле добавку, описываемую поляризуемостью \(\chi\):
\begin{align*}
	\vec{D} = \epsilon_0\epsilon \vec{E} = \epsilon_0(1 + \chi)\vec{E},
\end{align*}
при этом наличие какого-либо физического смысла у постоянной \(\epsilon_0\),
отсутствующей в системе СГС, является открытым вопросом.

Поляризуемость описывает поляризацию, т.\,е., поворот дипольных моментов:
\begin{align*}
	\epsilon_0 \chi \vec{E} = \vec{P} = \frac{1}{V}\sum_{i} \vec{p}_i.
\end{align*}

С точки зрения того, что выступает в роли этих дипольных моментов, 
можно выделить следующие виды поляризации:
\begin{itemize}
	\item дипольная: как правило, в данном случае речь идёт о веществах, состоящих из полярных молекул.
	Основной вклад в поляризацию в таких веществах обеспечивается за счёт поворота молекул вдоль поля.
	Время релаксации к поляризации в таких веществах составляет порядка \(10^{-10}~\text{с}\). Оценить его
	можно следующим образом: на поляризации воды работают микроволновые печи, следовательно, их частоту
	подбирают такой, чтобы молекулы в среднем успевали повернуться за полупериод волны (на соответствующей
	частоте должен наблюдаться максимум поглощения). Частота бытовой СВЧ-печи равна 2450~МГц. Следовательно, 
	\(\tau_{\text{H}_2\text{O}} \sim 1 / (2.45\cdot10^{9}~\text{с}^{-1}) = 4\cdot10^{-10}~\text{c}\);
	\item ионная: в данном случае речь идёт о кристаллах. В них дипольные моменты не могут поворачиваться,
	но ионы могут смещаться из положений равновесия в кристаллической решётке. Время релаксации порядка
	\(10^{-13}~\text{с}\);
	\item электронная: в роли дипольных моментов выступают атомы. Электрическое
	поле может деформировать их электронные оболочки. Время релаксации данного вида
	поляризации составляет порядка \(10^{-14}\)--\(10^{-15}\)~с.
\end{itemize}
Ориентация частиц, собственные частоты которых совпадают с частотой <<внешнего>> поля, называется резонансной поляризацией.

Помимо упомянутых типов поляризации, можно отметить спонтанную поляризацию (пироэлектричество),
сегнетоэлектричество, пьезоэлектричество\ldots

То же можно записать и для магнитной поляризации, но поворачиваться будут уже не дипольные моменты,
а магнитные:
\begin{align*}
	\vec{B} = \mu_0\mu\vec{H} = \mu_0(1 + &\gamma)\vec{H}, \\
	\mu_0&\gamma\vec{H} = \vec{M} = \frac{1}{c}\sum_i \vec{m}_i.
\end{align*}

\subsection{Замечание о киральных средах}
Можно предположить, что электрическая индукция зависит только от электрического поля,
а магнитная --- только от магнитного, однако это не так.%
\footnote{Получается, Бредов, Румянцев и Топтыгин неправы? Надо посмотреть, они, вроде, писали что-то на этот счёт.}
Примером сред, в которых такое описание не работает, являются киральные среды [здесь надо бы дать их определение и какой-нибудь обзор на четверть страницы.]

В частном случае киральных сред --- в диадных средах, --- материальные уравнения записываются следующим образом:
\begin{align*}
	\vec{D} &= \epsilon_0\epsilon E - i\eta\vec{H} \\
	\vec{H} &= \mu_0\mu\vec{B} + i\eta{B},
\end{align*}
где \(\eta\) называется элементом киральности.

Одно из особых свойств киральных сред демонстрируется в следующем простом эксперименте.
Прозрачный для радиоволн ящик заполняется левыми и правыми пружинами, в определённом соотношении.
К ящику подводится антенна или волновод, из которого приходит поляризованное(?) излучение от источника.
После прохода излучением ящика с пружинами, наблюдается поворот угла поляризации на угол, зависящий от соотношения
левых и правых пружин.


